\documentclass{beamer}

\input{settings.tex}


\title{Mixed Integer Convex Programming}
\subtitle{Computational Intelligence, Lecture 12}
\author{by Sergei Savin}
\centering
\date{\mydate}


\begin{document}
\maketitle


\begin{frame}{Content}

\begin{itemize}
\item Mixed Integer Linear Programming (MILP)
\item Mixed Integer Quadratic Programming (MIQP)
\item Example: Footstep planning
\item Big-M method relaxation
\item Example: switching control
\item Homework
\end{itemize}

\end{frame}



\begin{frame}{Mixed Integer Linear Programming (MILP)}
\framesubtitle{General form}
\begin{flushleft}

A general form of a mixed-integer linear program is:

%
\begin{equation} \label{LP}
\begin{aligned}
& \underset{\mathbf{x}}{\text{minimize}}
& & \mathbf{f}^\top \mathbf{x}, \\
& \text{subject to}
& & \begin{cases} 
\mathbf{A} \mathbf{x}
\leq 
\mathbf{b}, \\ 
\mathbf{C}\mathbf{x} = 
\mathbf{d},  \\
\mathbf{x}_1, ..., \mathbf{x}_m \in \mathbb{R},\\
\mathbf{x}_{m+1}, ..., \mathbf{x}_n \in \mathbb{N}.
\end{cases}
%
\end{aligned}
\end{equation}
 
In other words, the only difference is that some of the variables (denoted above as $\mathbf{y}$) are only allowed to assume pure integer values.
 
\end{flushleft}
\end{frame}



\begin{frame}{Mixed Integer Quadratic Programming (MIQP)}
\framesubtitle{General form}
\begin{flushleft}

A general form of a mixed-integer quadratic program is:

%
\begin{equation} \label{QP}
\begin{aligned}
& \underset{\mathbf{x}}{\text{minimize}}
& & 
\mathbf{x}^\top
\mathbf{H}
\mathbf{x} +
\mathbf{f}^\top
\mathbf{x} , \\
& \text{subject to}
& & \begin{cases} 
\mathbf{A} \mathbf{x}
\leq 
\mathbf{b}, \\ 
\mathbf{C}\mathbf{x} = 
\mathbf{d},  \\
\mathbf{x}_1, ..., \mathbf{x}_m \in \mathbb{R},\\
\mathbf{x}_{m+1}, ..., \mathbf{x}_n \in \mathbb{N}.
\end{cases}
%
\end{aligned}
\end{equation}
 
Other mixed-integer convex programs can be constructed likewise  from their pure convex counterparts.
 
\end{flushleft}
\end{frame}



\begin{frame}{Remarks}
% \framesubtitle{General form}
\begin{flushleft}

\begin{itemize}
    \item Mixed-integer convex programs are not convex, even if the name seems to suggest otherwise. 
    \item The "convex program" in the phrase "mixed-integer convex program" should be understood as "if we fix values of the integer variables, or relax them into reals, the result will be a convex program".
    \item Mixed integer programs are usually solved using \emph{branch and bound algorithms}; in worse case scenario, the algorithms performs exhaustive search.
    \item In robotics applications, integer variables in mixed integer programs are often restricted to binary values, i.e. $\mathbf{y} \in \{0, 1\}^n.$. It is still called "mixed-integer" in the literature, although phrases such as "mixed-binary" or "binary constraints" are not rare.
\end{itemize}
 
\end{flushleft}
\end{frame}



\begin{frame}{Example: Footstep planning}
\framesubtitle{Problem statement}
\begin{flushleft}

Given $N$ convex regions defined by linear inequalities $\{ \mathbf{x}: \  \mathbf{A}_i \mathbf{x} \leq \mathbf{b}_i \}$ (which is called H-polytope representation), find a sequence of $K$ points (footsteps) from the given starting point to the given goal point, such that all footsteps lie in one of the convex regions.

\begin{equation}
\begin{aligned}
& \underset{\mathbf{x}_1, \ ..., \ \mathbf{x}_K}{\text{minimize}}
& & ||\mathbf{x}_1 - \mathbf{x}_{\text{start}}|| + ||\mathbf{x}_K - \mathbf{x}_{\text{goal}}||, \\
& \text{subject to}
& & \exists \{ \mathbf{A}_j, \mathbf{b}_j\} \in \Omega \ \text{s.t.} \ \mathbf{A}_j\mathbf{x}_i \leq \mathbf{b}_j
\end{aligned}
\end{equation}

\bigskip

where $\Omega = \{ \{ \mathbf{A}_1, \mathbf{b}_1\}, \ \{ \mathbf{A}_2, \mathbf{b}_2\}, \ ..., \ \{ \mathbf{A}_N, \mathbf{b}_N\} \}$. This is not a convex program.
 
\end{flushleft}
\end{frame}




\begin{frame}{Big-M method relaxation}
\framesubtitle{Basic idea}
\begin{flushleft}


One of the key methods associated with the use of mixed-integer programming in robotics is the big-M method. 

\bigskip

Assume you have two inequalities, $\mathbf{a}_1^\top \mathbf{x} \leq b_1$ and $\mathbf{a}_2^\top \mathbf{x} \leq b_2$, and you are happy if at least one of them holds. Define a new binary variables $c_1, c_2 \in \{0, \ 1 \}$ and find a big enough constant $M$, such that $\mathbf{a}_1^\top \mathbf{x} \leq b_1 + M$ and $\mathbf{a}_2^\top \mathbf{x} \leq b_2 + M$ holds for all of your domain (of the part of it you are interested in). Then you write your constraints as:

\begin{equation}
    \begin{cases}
    \mathbf{a}_1^\top \mathbf{x} \leq b_1 + M \cdot c_1 \\
    \mathbf{a}_2^\top \mathbf{x} \leq b_2 + M \cdot c_2 \\
    c_1 + c_2 = 1  \\
    c_i  \in \{0, \ 1 \}
    \end{cases}
\end{equation}

\end{flushleft}
\end{frame}




\begin{frame}{Big-M method relaxation}
\framesubtitle{Systems of inequalities}
\begin{flushleft}

It works the same way for the case when you have three (or two or more) systems of inequalities $\mathbf{A}_1 \mathbf{x} \leq \mathbf{b}_1$, $\mathbf{A}_2 \mathbf{x} \leq \mathbf{b}_2$, $\mathbf{A}_3 \mathbf{x} \leq \mathbf{b}_3$ and are happy if at least one holds:

\begin{equation}
    \begin{cases}
    \mathbf{A}_1 \mathbf{x} \leq \mathbf{b}_1 + M \cdot \mathbf{1} \cdot (1 - c_1) \\
    \mathbf{A}_2 \mathbf{x} \leq \mathbf{b}_2 + M \cdot \mathbf{1} \cdot (1 - c_2) \\
    \mathbf{A}_3 \mathbf{x} \leq \mathbf{b}_3 + M \cdot \mathbf{1} \cdot (1 - c_3) \\
    c_1 + c_2 + c_3 = 1 \\
    c_i  \in \{0, \ 1 \}
    \end{cases}
\end{equation}

where $\mathbf{1}$ is a vector of all ones. Notice that constraint $c_1 + c_2 + c_3 = 1$ can be replaced with $c_1 + c_2 + c_3 >= 1$, allowing avoid relaxing more than one region.

\end{flushleft}
\end{frame}



\begin{frame}{Big-M method relaxation}
\framesubtitle{Illustration, part 1}
\begin{flushleft}

Below are two H-polytops (convex regions represented by systems of inequalities):

\begin{figure} [h!]
\begin{center}
\includegraphics[width=2.6in]{fig1.png}
\end{center} 
% \caption{AR 601 bipedal robot, Innopolis University}
\end{figure}

As we can see, their union represents a non-convex domain, and they have no intersection.

\end{flushleft}
\end{frame}


\begin{frame}{Big-M method relaxation}
\framesubtitle{Illustration, part 2}
\begin{flushleft}

Now, one of them is relaxed as described above. Notice that the intersection of the two polytopes is non-relaxed polytope. 

\begin{figure}
\centering
\begin{subfigure}{.5\textwidth}
  \centering
  \includegraphics[width=1.1\linewidth]{fig2.png}
%   \caption{Simplified model}
\end{subfigure}%
\begin{subfigure}{.5\textwidth}
  \centering
  \includegraphics[width=1.1\linewidth]{fig3.png}
%   \caption{Simplified model}
\end{subfigure}
\end{figure}

\end{flushleft}
\end{frame}


\begin{frame}{Big-M method relaxation}
\framesubtitle{Multiple variables}
\begin{flushleft}

If you have multiple variables $\mathbf{x}_1$, ..., $\mathbf{x}_K$, and each should belong to at least one of the H-polytopes $\{ \mathbf{A}_i, \mathbf{b}_i\}$, this can also be represented using big-M method:

\begin{equation}
    \begin{cases}
    \mathbf{A}_1 \mathbf{x}_k \leq \mathbf{b}_1 + M \cdot \mathbf{1} \cdot (1 - c_{1, k}) \\
    \mathbf{A}_2 \mathbf{x}_k \leq \mathbf{b}_2 + M \cdot \mathbf{1} \cdot (1 - c_{2, k}) \\
    \mathbf{A}_3 \mathbf{x}_k \leq \mathbf{b}_3 + M \cdot \mathbf{1} \cdot (1 - c_{3, k}) \\
    c_{1, k} + c_{2, k} + c_{3, k} = 1 \\
    c_{i, k}  \in \{0, \ 1 \} \\
    k = 1,...,K
    \end{cases}
\end{equation}

Notice that the only difference from the previous example is that now we have $K$ sets of binary variables $c_{1, k}$,  $c_{2, k}$ and $c_{3, k}$.

\end{flushleft}
\end{frame}






\begin{frame}{Example: Footstep planning}
\framesubtitle{Formulation as MIQP}
\begin{flushleft}

Using big-M relaxation we can now formulate the problem as follows:

\begin{equation}
\begin{aligned}
& \underset{\mathbf{x}_k, \ \mathbf{c}_{i,k}}{\text{minimize}}
& & ||\mathbf{x}_1 - \mathbf{x}_{\text{start}}|| + ||\mathbf{x}_K - \mathbf{x}_{\text{goal}}||, \\
& \text{subject to}
& & \begin{cases}
    \mathbf{A}_i \mathbf{x}_k \leq \mathbf{b}_i + M \cdot \mathbf{1} \cdot (1 - c_{i, k}), \ i = 1,...,N \\
    \sum_{i=1}^N c_{i, k} = 1 \\
    c  \in \{0, \ 1 \}^{N, \ K} \\
    k = 1,...,K
    \end{cases}
\end{aligned}
\end{equation}

 
\end{flushleft}
\end{frame}





\begin{frame}{Example: Footstep planning}
\framesubtitle{Evenly spaced steps}
\begin{flushleft}

In order to make the footsteps evenly spaced we add cost on the distance between consequent steps:

\begin{equation*}
\begin{aligned}
& \underset{\mathbf{x}_k, \ \mathbf{c}_{i,k}}{\text{minimize}}
& & ||\mathbf{x}_1 - \mathbf{x}_{\text{start}}||^2 + ||\mathbf{x}_K - \mathbf{x}_{\text{goal}}||^2 + w \cdot \sum_{k=1}^{K-1} ||\mathbf{x}_{i+1} - \mathbf{x}_i||^2, \\
& \text{subject to}
& & \begin{cases}
    \mathbf{A}_i \mathbf{x}_k \leq \mathbf{b}_i + M \cdot \mathbf{1} \cdot (1 - c_{i, k}), \ i = 1,...,N \\
    \sum_{i=1}^N c_{i, k} = 1 \\
    c  \in \{0, \ 1 \}^{N, \ K} \\
    k = 1,...,K
    \end{cases}
\end{aligned}
\end{equation*}

where $w$ is a weight, a coefficient we can tune to adjust relative importance of our primary objective (starting from the point $\mathbf{x}_{\text{start}}$ and finishing at the point $\mathbf{x}_{\text{goal}}$ and our secondary objective (making the footsteps evenly spaced).  
 
\end{flushleft}
\end{frame}




\begin{frame}{Example: Footstep planning}
\framesubtitle{Code, part 1}
\begin{flushleft}

\input{code1.tex}
 
\end{flushleft}
\end{frame}


\begin{frame}{Example: Footstep planning}
\framesubtitle{Code, part 2}
\begin{flushleft}

\input{code2.tex}
 
\end{flushleft}
\end{frame}


\begin{frame}{Example: Footstep planning}
\framesubtitle{Code, part 3}
\begin{flushleft}

\begin{lstlisting}[language=Matlab]
n = 5; m = 2;
A = randn(n, n);
B = randn(n, m);
Q = eye(n)*0.1;
cvx_begin sdp
    variable P(n, n) symmetric
    variable L(m, n)
    
    minimize 0
    subject to
    	P >= 0;
        A*P + P*A' + B*L + L'*B' + Q <= 0;
cvx_end
P = full(P);
L = full(L);
K = L*pinv(P);

disp('CL eig:')
eig(A + B*K)
\end{lstlisting}
 
\end{flushleft}
\end{frame}



\begin{frame}{Example: switching control}
	\framesubtitle{Multiple variables}
	\begin{flushleft}
		
		Consider the case when you have a control system $\dot{\bo{x}} =\bo{A}\bo{x} + \bo{B}_1\bo{u}_1 + \bo{B}_2\bo{u}_2$, but you are required to use either $\bo{B}_1\bo{u}_1$ or $\bo{B}_2\bo{u}_2$ at any time, not both.
		
		\bigskip
		
		This can be cast as a mixed-integer constraint:
		
		\begin{equation}
			\begin{cases}
				||\bo{B}_1\bo{u}_1|| \leq M (1 - c_1) \\
				||\bo{B}_2\bo{u}_2|| \leq M (1 - c_2) \\
				c_1 + c_2 = 1
			\end{cases}
		\end{equation}
		
	\end{flushleft}
\end{frame}




\begin{frame}{Homework}
% \framesubtitle{Parameter estimation}
\begin{flushleft}

Implement footstep planning for a biped, making sure every pair of steps lands in the same H-polytope.

\end{flushleft}
\end{frame}



\begin{frame}
	\centerline{Lecture slides are available via Moodle.}
	\bigskip
	\centerline{You can help improve these slides at:}
	\centerline{
		\mygit
	}
	\bigskip
	
	\textcolor{black}{\qrcode[height=1.5in]{https://github.com/SergeiSa/Computational-Intelligence-Slides-Spring-2022}}
	\bigskip
	
	
	\centerline{Check Moodle for additional links, videos, textbook suggestions.}
\end{frame}

\end{document}
