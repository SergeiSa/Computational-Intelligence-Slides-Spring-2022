\documentclass{beamer}

\input{settings.tex}


\title{Linear Programming}
\subtitle{Computational Intelligence, Lecture 8}
\author{by Sergei Savin}
\centering
\date{Spring 2021}

\begin{document}
\maketitle


\begin{frame}{Content}

\begin{itemize}
\item Linear Programming
\begin{itemize}
    \item General form
    \item LP with no solution - examples
\end{itemize}
\item Convex piece-wise linear functions
\begin{itemize}
    \item Problem statement
    \item Solution as LP
    \item Sum of piece-wise linear functions
    \item Code example
\end{itemize}
\item Chebyshev center of a polyhedron
\begin{itemize}
    \item Problem statement
    \item Solution as LP
    \item Code example
\end{itemize}
\item Homework
\end{itemize}

\end{frame}



\begin{frame}{Linear Programming}
\framesubtitle{General form}
\begin{flushleft}

A linear program (LP) is an optimization problem of the form:

\begin{equation} \label{LP}
\begin{aligned}
& \underset{\mathbf{x}}{\text{minimize}}
& & \mathbf{f}^\top \mathbf{x} , \\
& \text{subject to}
& & \begin{cases} 
\mathbf{A}\mathbf{x} \leq \mathbf{b}, \\
\mathbf{C}\mathbf{x} = \mathbf{d}.
\end{cases}
%
\end{aligned}
\end{equation}

It is one of the older and widely used classes of convex optimization problems. 

\bigskip

Note that the solution of such problem will always lie on the boundary of its domain.
 
\end{flushleft}
\end{frame}




\begin{frame}{Linear Programming}
\framesubtitle{LP with no solution - examples}
\begin{flushleft}

Here are some examples of LP which have no solutions:

\begin{equation}
\begin{aligned}
& \underset{\mathbf{x}}{\text{minimize}}
& & \begin{bmatrix} 1 & 1 \end{bmatrix} 
\begin{bmatrix} x_1 \\ x_2 \end{bmatrix}
\end{aligned}
\end{equation}

This one is has no boundaries at all, hence no solution. Next one has boundaries, but they do not restrict motion along the descent direction for the cost function.

\begin{equation}
\begin{aligned}
& \underset{\mathbf{x}}{\text{minimize}}
& & \begin{bmatrix} 1 & 1 \end{bmatrix} 
\begin{bmatrix} x_1 \\ x_2 \end{bmatrix} , \\
& \text{subject to}
& & \begin{bmatrix} 1 & 0 \end{bmatrix}
\begin{bmatrix} x_1 \\ x_2 \end{bmatrix} \leq
1
%
\end{aligned}
\end{equation}

 
\end{flushleft}
\end{frame}



\begin{frame}{Convex piece-wise linear functions}
\framesubtitle{Problem statement}
\begin{flushleft}

Convex piece-wise linear functions have the form:

\begin{equation}
    f(\mathbf{x}) = \text{max}(\mathbf{a}_i^\top \mathbf{x} + b_i)
\end{equation}

Figure below shows geometric interpretation of such function for a one-dimensional case.

\begin{figure} [h!]
\begin{center}
\input{fig_1.tex}
\end{center} 
% \caption{Visualization of trajectory generation done in the developed software}
\end{figure}

 
\end{flushleft}
\end{frame}




\begin{frame}{Convex piece-wise linear functions}
\framesubtitle{Solution as LP}
\begin{flushleft}

We can formulate a minimization problem using convex piece-wise linear functions:

\begin{equation}
\begin{aligned}
& \underset{\mathbf{x}}{\text{minimize}}
& & \text{max}(\mathbf{a}_i^\top \mathbf{x} + b_i)
\end{aligned}
\end{equation}

\bigskip

Which can be equivalently transformed into the following LP:

\begin{equation}
\begin{aligned}
& \underset{\mathbf{x}, t}{\text{minimize}}
& & t \\
& \text{subject to}
& & \mathbf{a}_i^\top \mathbf{x} + b_i \leq t
%
\end{aligned}
\end{equation}

We can observe that optimal (minimal) $t$ will have to lie on one of the linear functions $\mathbf{a}_i^\top \mathbf{x} + b_i$, i.e. on the original piece-wise linear function $f(\mathbf{x})$. And optimal value on t corresponds to the smallest value of the original function $f(\mathbf{x})$.
 
\end{flushleft}
\end{frame}



\begin{frame}{Sum of piece-wise linear functions}
\framesubtitle{Solution as LP}
\begin{flushleft}


Sum of convex piece-wise linear functions have the form:

\begin{equation}
    f(\mathbf{x}) + g(\mathbf{x}) = \text{max}(\mathbf{a}_i^\top \mathbf{x} + b_i) +  \text{max}(\mathbf{c}_i^\top \mathbf{x} + d_i)
\end{equation}

\bigskip

Their representation as LP is:

\begin{equation}
\begin{aligned}
& \underset{\mathbf{x}, t_1, t_2}{\text{minimize}}
& & t_1 + t_2 \\
& \text{subject to}
& & \begin{cases}
\mathbf{a}_i^\top \mathbf{x} + b_i \leq t_1 \\
\mathbf{c}_i^\top \mathbf{x} + d_i \leq t_2
\end{cases}
%
\end{aligned}
\end{equation}


 
\end{flushleft}
\end{frame}




\begin{frame}{Convex piece-wise linear functions}
\framesubtitle{Code}
\begin{flushleft}

\input{code1}
 
\end{flushleft}
\end{frame}



\begin{frame}{Chebyshev center of a polyhedron}
\framesubtitle{Problem statement}
\begin{flushleft}

Chebyshev center of a polyhedron is the center of the largest ball inscribed in a polyhedron:

\begin{figure} [h!]
\begin{center}
\input{fig_2.tex}
\end{center} 
% \caption{Visualization of trajectory generation done in the developed software}
\end{figure}

Equation describing this ball can be written as:

\begin{equation}
    \mathcal{B} = \{ \mathbf{x}_c + \mathbf{u}: \ ||\mathbf{u}||_2 \leq r \}
\end{equation}

where $r$ is the radius of the ball and $\mathbf{x}_c$ is its center.
 
\end{flushleft}
\end{frame}


\begin{frame}{Chebyshev center of a polyhedron}
\framesubtitle{Solution as LP, part one}
\begin{flushleft}

For the ball $\mathcal{B}$ to be inscribed in a polygon $\mathcal{P} = \{ \mathbf{x}: \ \mathbf{A}\mathbf{x} \leq \mathbf{b} \}$, the following should hold:

\begin{equation}
    \text{sup} \{ \mathbf{a}_i^\top (\mathbf{x}_c + \mathbf{u}): \ ||\mathbf{u}||_2 \leq r \} \leq b_i
\end{equation}

Note that the largest value of $\mathbf{a}_i^\top \mathbf{u}$ under condition $||\mathbf{u}||_2 \leq r$ is $r ||\mathbf{a}_i||$: it can indeed achieve this value if $\mathbf{a}_i$ and $\mathbf{u}$ are co-directional, but a larger one is not possible. Therefore:

\begin{equation}
    \text{sup} \{ \mathbf{a}_i^\top (\mathbf{x}_c + \mathbf{u}): \ ||\mathbf{u}||_2 \leq r \}  = 
    \mathbf{a}_i^\top \mathbf{x}_c + r ||\mathbf{a}_i||
    \leq b_i
\end{equation}

 
\end{flushleft}
\end{frame}



\begin{frame}{Chebyshev center of a polyhedron}
\framesubtitle{Solution as LP, part two}
\begin{flushleft}

Finally, we can write down the solution of the problem as a linear optimization:

\begin{equation}
\begin{aligned}
& \underset{r, \ \mathbf{x}_c}{\text{maximize}}
& & r \\
& \text{subject to}
& & \mathbf{a}_i^\top \mathbf{x}_c + r ||\mathbf{a}_i||
    \leq b_i
%
\end{aligned}
\end{equation}

 
\end{flushleft}
\end{frame}




\begin{frame}{Chebyshev center of a polyhedron}
\framesubtitle{Code}
\begin{flushleft}

Below we can see MATLAB code for solving the problem:

\input{code2.tex}

 
\end{flushleft}
\end{frame}



\begin{frame}{Homework}
% \framesubtitle{Parameter estimation}
\begin{flushleft}

Implement linear approximation of a convex function and solve it as LP

\end{flushleft}
\end{frame}



\begin{frame}
\centerline{Lecture slides are available via Moodle.}
\bigskip
\centerline{You can help improve these slides at:}
\centerline{
\textcolor{blue}{\href{https://github.com/SergeiSa/Computational-Intelligence-Slides-Spring-2021}{github.com/SergeiSa/Computational-Intelligence-Slides-Spring-2021}}
}
\bigskip

\textcolor{black}{\qrcode[height=1.5in]{https://git.io/JYRBT}}
\bigskip

\centerline{Check Moodle for additional links, videos, textbook suggestions.}
\end{frame}

\end{document}

\end{document}
