\documentclass{beamer}

\pdfmapfile{+sansmathaccent.map}


\mode<presentation>
{
  \usetheme{Warsaw} % or try Darmstadt, Madrid, Warsaw, Rochester, CambridgeUS, ...
  \usecolortheme{crane} % or try seahorse, beaver, crane, wolverine, ...
  \usefonttheme{serif}  % or try serif, structurebold, ...
  \setbeamertemplate{navigation symbols}{}
  \setbeamertemplate{caption}[numbered]
} 


%%%%%%%%%%%%%%%%%%%%%%%%%%%%
% itemize settings

\definecolor{myhotpink}{RGB}{255, 80, 200}
\definecolor{mywarmpink}{RGB}{255, 60, 160}
\definecolor{mylightpink}{RGB}{255, 80, 200}
\definecolor{mypink}{RGB}{255, 30, 80}
\definecolor{mydarkpink}{RGB}{155, 25, 60}
\definecolor{myblue}{RGB}{240, 240, 255}
\definecolor{mydarkblue}{RGB}{60, 160, 255}
\definecolor{mygreen}{RGB}{0, 200, 0}
\definecolor{mygreen2}{RGB}{245, 255, 230}
\definecolor{mygray}{gray}{0.8}


\definecolor{mydarkcolor}{RGB}{60, 25, 155}
\definecolor{mylightcolor}{RGB}{130, 180, 250}

\setbeamertemplate{itemize items}[default]

\setbeamertemplate{itemize item}{\color{mywarmpink}$\blacksquare$}
\setbeamertemplate{itemize subitem}{\color{mydarkblue}$\blacktriangleright$}
\setbeamertemplate{itemize subsubitem}{\color{mygray}$\blacksquare$}



\setbeamercolor{palette quaternary}{fg=white,bg=mydarkcolor}
\setbeamercolor{titlelike}{parent=palette quaternary}

\setbeamercolor{palette quaternary2}{fg=black,bg=mylightcolor}
\setbeamercolor{frametitle}{parent=palette quaternary2}



\setbeamerfont{frametitle}{size=\Large,series=\scshape}
\setbeamerfont{framesubtitle}{size=\normalsize,series=\upshape}


%%%%%%%%%%%%%%%%%%%%%%%%%%%%
% block settings

\setbeamercolor{block title}{bg=red!50,fg=black}

\setbeamercolor*{block title example}{bg=mygreen!40!white,fg=black}

\setbeamercolor*{block body example}{fg= black,
bg= mygreen2}


%%%%%%%%%%%%%%%%%%%%%%%%%%%%
% URL settings
\hypersetup{
    colorlinks=false,
    linkcolor=blue,
    filecolor=blue,      
    urlcolor=blue,
}

%%%%%%%%%%%%%%%%%%%%%%%%%%

\renewcommand{\familydefault}{\rmdefault}

\usepackage{amsmath}
\usepackage{mathtools}

\usepackage{subcaption}


\newcommand{\mydate}{Spring 2022}
\newcommand{\mygit}{\textcolor{blue}{\href{https://github.com/SergeiSa/Computational-Intelligence-Slides-Spring-2022}{github.com/SergeiSa/Computational-Intelligence-Slides-Spring-2022}}}

\newcommand{\bo}[1] {\mathbf{#1}}
\newcommand{\R} {\mathbb{R}}
\DeclareMathOperator*{\argmin}{arg\,min}


%%%%%%%%%%%%%%%%%%%%%%%%%%%%
% code settings

\usepackage{listings}
\usepackage{color}
% \definecolor{mygreen}{rgb}{0,0.6,0}
% \definecolor{mygray}{rgb}{0.5,0.5,0.5}
\definecolor{mymauve}{rgb}{0.58,0,0.82}
\lstset{ 
  backgroundcolor=\color{white},   % choose the background color; you must add \usepackage{color} or \usepackage{xcolor}; should come as last argument
  basicstyle=\footnotesize,        % the size of the fonts that are used for the code
  breakatwhitespace=false,         % sets if automatic breaks should only happen at whitespace
  breaklines=true,                 % sets automatic line breaking
  captionpos=b,                    % sets the caption-position to bottom
  commentstyle=\color{mygreen},    % comment style
  deletekeywords={...},            % if you want to delete keywords from the given language
  escapeinside={\%*}{*)},          % if you want to add LaTeX within your code
  extendedchars=true,              % lets you use non-ASCII characters; for 8-bits encodings only, does not work with UTF-8
  firstnumber=0000,                % start line enumeration with line 0000
  frame=single,	                   % adds a frame around the code
  keepspaces=true,                 % keeps spaces in text, useful for keeping indentation of code (possibly needs columns=flexible)
  keywordstyle=\color{blue},       % keyword style
  language=Octave,                 % the language of the code
  morekeywords={*,...},            % if you want to add more keywords to the set
  numbers=left,                    % where to put the line-numbers; possible values are (none, left, right)
  numbersep=5pt,                   % how far the line-numbers are from the code
  numberstyle=\tiny\color{mygray}, % the style that is used for the line-numbers
  rulecolor=\color{black},         % if not set, the frame-color may be changed on line-breaks within not-black text (e.g. comments (green here))
  showspaces=false,                % show spaces everywhere adding particular underscores; it overrides 'showstringspaces'
  showstringspaces=false,          % underline spaces within strings only
  showtabs=false,                  % show tabs within strings adding particular underscores
  stepnumber=2,                    % the step between two line-numbers. If it's 1, each line will be numbered
  stringstyle=\color{mymauve},     % string literal style
  tabsize=2,	                   % sets default tabsize to 2 spaces
  title=\lstname                   % show the filename of files included with \lstinputlisting; also try caption instead of title
}

%%%%%%%%%%%%%%%%%%%%%%%%%%%%
% tikz settings

\usepackage{tikz}
\tikzset{every picture/.style={line width=0.75pt}}

%%%%%%%%%%%%%%%%%%%%%%%%%%%%

\usepackage{qrcode}



\title{Least Squares, Null space, Row space, Projectors}
\subtitle{Computational Intelligence, Lecture 2}
\author{by Sergei Savin}
\centering
\date{\mydate}



\begin{document}
\maketitle


\begin{frame}{Content}

\begin{itemize}
\item Motivating questions
\item Four Fundamental Subspaces
\item Null space
\begin{itemize}
    \item Definition
    \item Calculation
\end{itemize}
\item Null space projection
\item Closest element from a linear subspace
\item Orthogonality, definition
\item Projection
\item Vectors in Null space, Row space
\item Row and Null spaces in linear equations
% \item Read more
\end{itemize}

\end{frame}




\begin{frame}{Least squares at a glance}
	% \framesubtitle{Parameter estimation}
	\begin{flushleft}
		
	Consider the following problem: find smallest-norm $\bo{x}$ that equality $\bo{A}\bo{x} = \bo{y}$ has least residual. This is the \emph{least squares problem}.
	
	\bigskip
	
	Solution to the least squares problem is given by a pseudoinverse:
	
	\begin{equation}
		\bo{x} = \bo{A}^+\bo{y}
	\end{equation}

Notice that, surprisingly, this solves both minimizations at the same time: find smallest $\bo{x}$ among all least-residual solutions.
		
	\end{flushleft}
\end{frame}



\begin{frame}{Least squares and closest element}
	% \framesubtitle{Parameter estimation}
	\begin{flushleft}
		
		You are given equation $\bo{A}\bo{x} = \bo{y}$. Assume that you want to find such $\bo{x}$ that $ \bo{y}$ achieves the value as close as possible to $\bo{y}^*$.
		
		\bigskip
		
		We know that $\bo{x} = \bo{A}^+\bo{y}^*$ gives us the least residual solution. Multiplying it by $\bo{A}$ we get:
		
		\begin{equation}
			\bo{y} = \bo{A}\bo{A}^+\bo{y}^*
		\end{equation}
	
	This is the value of $\bo{y}$ closest to $\bo{y}^*$, that we can achieve.
	
	\end{flushleft}
\end{frame}



%\begin{frame}{Motivating questions}
%% \framesubtitle{Parameter estimation}
%\begin{flushleft}
%
%You have a linear operator $\mathbf A$. Try to answer the following questions:
%
%\begin{itemize}
%    \item What are all vectors this operator can produce as outputs? How to find them?
%    \item Are there two inputs that make it produce the same output?
%    \item Are there inputs that produce zero as an output?
%    \item Are there outputs that cannot be produced? 
%    \item What is the smallest vector $\bo{x}$ that produces given output $\bo{y}$? 
%\end{itemize}
%
%These questions are directly related to the idea of fundamental subspaces of a linear operator.
%
%\end{flushleft}
%\end{frame}


\begin{frame}{Four Fundamental Subspaces}
% \framesubtitle{Parameter estimation}
\begin{flushleft}

One of the key ideas in the linear algebra is that every linear operator has four fundamental subspaces:

\begin{itemize}
    \item Null space
    \item Row space
    \item Column space
    \item Left null space
\end{itemize}

\bigskip

Our goal is to understand them. The usefulness of this understating is enormous.

\end{flushleft}
\end{frame}

\begin{frame}{Null space}
\framesubtitle{Definition}
\begin{flushleft}

Consider the following task: find all solutions to the system of equations $\mathbf{A} \mathbf{x} = \mathbf{0}$.

\bigskip

It can be re-formulated as follows: find all elements of the \emph{null space} of $\mathbf{A}$.

\begin{block}{Definition 1}
  \emph{Null space} of $\mathbf{A}$ is the set of all vectors $\mathbf{x}$ that $\mathbf{A}$ maps to $\mathbf{0}$
\end{block}

\bigskip

We will denote null space as $\mathcal{N}(\mathbf{A})$. In the literature, it is often denoted as $\text{ker}(\mathbf{A})$ or $\text{null}(\mathbf{A})$.

\end{flushleft}
\end{frame}


\begin{frame}{Null space}
\framesubtitle{Calculation}
\begin{flushleft}

Now we can find all solutions to the system of equations $\mathbf{A} \mathbf{x} = \mathbf{0}$ by using functions that generate an orthonormal \emph{basis} in the null space of $\mathbf{A}$. In MATLAB it is function \texttt{null}, in Python/Scipy - \texttt{null\_space}:

\bigskip

\begin{itemize}
    \item \texttt{N = null(A)}.
    \item \texttt{N = scipy.linalg.null\_space(A)}.
\end{itemize}

\bigskip

That is it! Space of solutions of $\mathbf{A} \mathbf{x} = \mathbf{0}$ is the span of the columns of $\mathbf{N}$, and all solutions $\mathbf{x}^*$ can be represented as $\mathbf{x}^* = \mathbf{N}\mathbf{z}$; for any $\mathbf{z}$ we get a unique solution, and for any solution - a unique $\mathbf{z}$.

\end{flushleft}
\end{frame}



\begin{frame}{Null space projection}
\framesubtitle{Local coordinates}
\begin{flushleft}

Let $\bo{N}$ be the orthonormal basis in the null space of matrix $\bo{A}$. Then, if a vector $\bo{x}$ lies in the null space of $\bo{A}$, it can be represented as:

\begin{equation}
    \bo{x} = \bo{N}\bo{z}
\end{equation}
%
where $\bo{z}$ are coordinates of $\bo{x}$ in the basis $\bo{N}$.

\bigskip

However, there are vector which not only are not lying in the null space of $\bo{A}$,  but the closest vector to them in the null space is zero vector.

\end{flushleft}
\end{frame}


\begin{frame}{Closest element from a linear subspace}
% \framesubtitle{Orthogonality, examples}
\begin{flushleft}

Let $\bo{A} = \begin{bmatrix} 0 & 1 \\ 0 & 0\end{bmatrix}$. Its null space has orthonormal basis $\bo{N} = \begin{bmatrix} 1 \\ 0\end{bmatrix}$. 

\begin{itemize}
    \item $\begin{bmatrix} -2 \\ 0 \end{bmatrix} = 
    -2 \bo{N}$,
    $\begin{bmatrix} 10 \\ 0 \end{bmatrix} = 
    10 \bo{N}$, - both are in the null space.
    \item for $\bo{x} = \begin{bmatrix} 1 \\ 1 \end{bmatrix}$ the closest vector in the null space is $\begin{bmatrix} 1 \\ 0 \end{bmatrix}$.
    \item for $\bo{y} = \begin{bmatrix} 0 \\ 2 \end{bmatrix}$ the closest vector in the null space is $\begin{bmatrix} 0 \\ 0 \end{bmatrix}$.
\end{itemize}


\end{flushleft}
\end{frame}



\begin{frame}{Orthogonality, definition}
% \framesubtitle{Orthogonality, definition}
\begin{flushleft}

\begin{definition}
If for a vector $\bo{x}$, the closest vector to it from a linear subspace $\mathcal{L}$ is zero vector, $\bo{x}$ is called \emph{orthogonal} to the subspace $\mathcal{L}$. We denote it as $\bo{x} \in \mathcal{L}^\perp$.
\end{definition}

\begin{definition}
	(equivalent) A vector $\bo{x}$, orthogonal to all elements of the subspace $\mathcal{L}$ is called \emph{orthogonal} to the subspace $\mathcal{L}$.
\end{definition}

\begin{definition}
The space of all vectors $\bo{x}$, orthogonal to a linear subspace $\mathcal{L}$ is called \emph{orthogonal compliment} of $\mathcal{L}$ and is denoted as $\mathcal{L}^\perp$.
\end{definition}

\end{flushleft}
\end{frame}





\begin{frame}{Projection}
\framesubtitle{Part 1}
\begin{flushleft}

Let $\bo{L}$ be an orthonormal basis in a linear subspace $\mathcal{L}$. Take vector $\bo{a} = \bo{x} + \bo{y}$, where $\bo{x}$ lies in the subspace $\mathcal{L}$, and $\bo{y}$ is orthogonal to $\mathcal{L}$.

\bigskip

\begin{definition}
We call such vector $\bo{x}$ a \emph{projection} of $\bo{a}$ onto subspace $\mathcal{L}$, and such vector $\bo{y}$ a projection of $\bo{a}$ onto subspace $\mathcal{L}^\perp$
\end{definition}

\bigskip

Projection of $\bo{a}$ onto $\mathcal{L}$ can be found as: 

\begin{equation}
    \bo{x} = \bo{L} \bo{L}^+ \bo{a}
\end{equation}

Since $\bo{L}$ is orthonormal, this is the same as $\bo{x} = \bo{L} \bo{L}^\top \bo{a}$

\end{flushleft}
\end{frame}



\begin{frame}{Projection}
\framesubtitle{Part 2}
\begin{flushleft}

Since $\bo{a} = \bo{x} + \bo{y}$, and $\bo{x} = \bo{L} \bo{L}^\top \bo{a}$, we can write:

\begin{equation}
    \bo{a} = \bo{L} \bo{L}^\top \bo{a} + \bo{y}
\end{equation}
%
from which it follows that the projection of $\bo{a}$ onto $\mathcal{L}^\perp$ can be found as: 

\begin{equation}
    \bo{y} = (\bo{I} - \bo{L} \bo{L}^\top) \bo{a}
\end{equation}
%
where $\bo{I}$ is an identity matrix. Since $\bo{L}$ is orthonormal, this is the same as $\bo{y} = (\bo{I} - \bo{L} \bo{L}^\top) \bo{a}$

\end{flushleft}
\end{frame}




\begin{frame}{Row space}
\framesubtitle{Definition}
\begin{flushleft}

\begin{definition}
Let $\mathcal{N}$ be null space of $\bo{A}$. Then orthogonal subspace $\mathcal{N}^\perp$ is called \emph{row space} of $\bo{A}$.
\end{definition}

\begin{definition}
\emph{Row space} of $\bo{A}$ is the space of all smallest-norm solutions of $\bo{A}\bo{x} = \bo{y}$, for $\forall \bo{y}$, plus the zero vector, which is included in all linear subspaces.
\end{definition}

\bigskip

We will denote row space as $\mathcal{R}$.

\end{flushleft}
\end{frame}




\begin{frame}{Vectors in Null space, Row space}
% \framesubtitle{Definition}
\begin{flushleft}

Given vector $\bo{x}$, matrix $\bo{A}$ and its nulls space basis $\bo{N}$, and we check if $\bo{x}$ is in the null space of $\bo{A}$. The simplest way is to check if $\bo{A}\bo{x} = 0$. But sometimes we may want to avoid computing $\bo{A}\bo{x}$, for example if the number of elements of $\bo{A}$ is much bigger than the number of elements of $\bo{N}$.

\bigskip

We notice that if $\bo{x}$ is in the null space of $\bo{A}$, it will have zero projection onto the row space of $\bo{A}$. So, the condition is as follows:

\begin{equation}
    (\bo{I} - \bo{N} \bo{N}^\top) \bo{x} = 0
\end{equation}

By the same logic, condition for being in the row space is as follows:

\begin{equation}
    \bo{N} \bo{N}^\top \bo{x} = 0
\end{equation}


\end{flushleft}
\end{frame}







\begin{frame}{Row and Null spaces in linear equations}
\framesubtitle{Part 1}
\begin{flushleft}

Consider another task: find all solutions to the system of equations $\bo{A} \bo{x} = \bo{y}$.

\bigskip

Assume we have two solutions to the system: $\bo{x}_1$ and $\bo{x}_2$. We know that $\bo{A} \bo{x}_1 = \bo{A} \bo{x}_2= \bo{y}$, hence $\bo{A} (\bo{x}_1 - \bo{x}_2) = \bo{0}$. In other words, the difference between any two solutions lies in the null space of $\bo{A}$.

\bigskip

On the other hand, let $\bo{x}^*$ be a solution, and $\bo{x}^N \in \mathcal{N}(\bo{A})$ be a vector in the null space of $\bo{A}$. Then $\bo{x}^* + \bo{x}^N$ is also a solution, since $\bo{A} (\bo{x}^* + \bo{x}^N) = \bo{A} \bo{x}^* + \bo{A} \bo{x}^N = \bo{A} \bo{x}^* = \bo{y}$.

\bigskip

Therefore, the solution space is given by a single partial solution $\bo{x}^p \notin \mathcal{N}(\bo{A})$ and the whole null space of $\bo{A}$.

\end{flushleft}
\end{frame}


\begin{frame}{Row and Null spaces in linear equations}
\framesubtitle{Part 2}
\begin{flushleft}

There are infinitely many ways to chose $\bo{x}^p$, since if $\bo{x}^p \notin \mathcal{N}(\bo{A})$, then $(\bo{x}^p + \bo{x}^N) \notin \mathcal{N}(\bo{A})$, if $\bo{x}^N \in \mathcal{N}(\bo{A})$. However: 

\begin{block}{Statement 1}
The smallest-norm $\bo{x}^p$ will lie in the row space of $\bo{A}$.
\end{block}

\bigskip

We can prove it by observing that there can be only one $\bo{x}^p \in \mathcal{R}(\bo{A})$ and adding to it any vector $\bo{x}^N \in \mathcal{N}(\bo{A})$ can only increase its magnitude, as $\bo{x}^p$ and $\bo{x}^N$ are orthogonal.

\end{flushleft}
\end{frame}



\begin{frame}{Row and Null spaces in linear equations}
\framesubtitle{Part 3}
\begin{flushleft}

If we have $\bo{x}^*$, which is a solution to $\bo{A} \bo{x} = \bo{y}$, we can find the particular solution $\bo{x}^p \in \mathcal{R}(\bo{A})$ as a projection:

\begin{equation}
    \bo{x}^p = (\bo{I} - \bo{N} \bo{N}^\top) \bo{x}^*
\end{equation}
%
where $\bo{N}$ is the null space basis for $\bo{A}$. Alternatively, we can simply find it as:

\begin{equation}
    \bo{x}^p = \bo{A}^+ \bo{y}
\end{equation}

\bigskip

All solutions to $\bo{A} \bo{x} = \bo{y}$ are then given as:

\begin{equation}
    \bo{x}^* = \bo{A}^+ \bo{y} + \bo{N}\bo{z}, \ \forall \bo{z}
\end{equation}

\end{flushleft}
\end{frame}




\begin{frame}
	\centerline{Lecture slides are available via Moodle.}
	\bigskip
	\centerline{You can help improve these slides at:}
	\centerline{
		\mygit
	}
	\bigskip
	
	\textcolor{black}{\qrcode[height=1.5in]{https://github.com/SergeiSa/Computational-Intelligence-Slides-Spring-2022}}
	\bigskip
	
	
	\centerline{Check Moodle for additional links, videos, textbook suggestions.}
\end{frame}

\end{document}
