\documentclass{beamer}

\input{settings.tex}


\title{Barrier functions}
\subtitle{Computational Intelligence, Lecture 11}
\author{by Sergei Savin}
\centering
\date{Spring 2021}



\begin{document}
\maketitle


\begin{frame}{Content}

\begin{itemize}
\item Linear inequalities
\item Barrier functions
\item Barrier functions for QPs
\item Analytic center of linear inequalities
\item Homework
\end{itemize}

\end{frame}



\begin{frame}{Linear inequalities}
% \framesubtitle{General form}
\begin{flushleft}

Consider linear inequality constraints:

\begin{equation}
    \bo{A}\bo{x} \leq \bo{b}
\end{equation}

Remember that we can rewrite it as:

\begin{equation}
    \bo{a}_i^\top \bo{x} \leq b_i
\end{equation}
\begin{equation}
\label{eq:linear_constraints}
    \bo{a}_i^\top \bo{x} - b_i \leq 0
\end{equation}

Instead of \emph{hard constraints} in \eqref{eq:linear_constraints} we can turn these into a cost function component:

\begin{equation}
    J = -\sum\limits_{i = 1}^n \text{log} (b_i - \bo{a}_i^\top \bo{x})
\end{equation}

Which is called a \emph{barrier function}.
 
\end{flushleft}
\end{frame}




\begin{frame}{Barrier functions}
% \framesubtitle{General form}
\begin{flushleft}

Let us consider barrier functions $J = -\sum\limits_{i = 1}^n \text{log} (b_i - \bo{a}_i^\top \bo{x})$:

\begin{itemize}
    \item It removes the constraint, but modifies the cost.
    \item When $b_i - \bo{a}_i^\top \bo{x}$ is a very small positive number, $\text{log} (b_i - \bo{a}_i^\top \bo{x})$ is a very big negative number, hence the minus sign in front.
    \item Barrier function does not behave well outside of the domain, when $b_i - \bo{a}_i^\top \bo{x} < 0$.
\end{itemize}
 
\end{flushleft}
\end{frame}




\begin{frame}{Barrier functions for QPs}
% \framesubtitle{General form}
\begin{flushleft}

Hence the following QP:

\begin{equation}
\begin{aligned}
& \underset{\bo{x}}{\text{minimize}}
& & \bo{x}^\top \bo{H} \bo{x} + \bo{f}^\top \bo{x}, \\
& \text{subject to}
& & \begin{cases}
    \bo{A}\bo{x} \leq \bo{b}, \\
    \bo{C}(\bo{x}) = \bo{d}.
    \end{cases}
\end{aligned}
\end{equation}
 
...can be approximated as:

\begin{equation}
\begin{aligned}
& \underset{\bo{x}}{\text{minimize}}
& & \bo{x}^\top \bo{H} \bo{x} + \bo{f}^\top \bo{x} - \sum\limits_{i = 1}^n \text{log} (b_i - \bo{a}_i^\top \bo{x}), \\
& \text{subject to}
& & \bo{C}(\bo{x}) = \bo{d}
\end{aligned}
\end{equation}
 
\end{flushleft}
\end{frame}




\begin{frame}{Analytic center of linear inequalities}
% \framesubtitle{General form}
\begin{flushleft}

We can define \emph{analytic center of linear inequalities} as a minimum of the function $J = -\sum\limits_{i = 1}^n \text{log} (b_i - \bo{a}_i^\top \bo{x})$. And that can be solved as a convex optimization:

\begin{align*}
    \bo{x}_a = \underset{\bo{x}}{\text{argmin}} & \ \  -\sum\limits_{i = 1}^n \text{log} (b_i - \bo{a}_i^\top \bo{x})
\end{align*}

At the analytic center of linear inequalities the shape of contour lines can be analysed as a local quadratic approximation of the function $J$:

\begin{equation}
    \mathcal{C} = \{ \bo{x}: \ (\bo{x} - \bo{x}_a)^\top \frac{\partial^2 J}{\partial \bo{x}^2} (\bo{x} - \bo{x}_a) = \epsilon \}
\end{equation}

where $\epsilon$ is a small number.
  
\end{flushleft}
\end{frame}



\begin{frame}{Illustration of a barrier functions}
% \framesubtitle{Parameter estimation}
\begin{flushleft}

\begin{figure}
    \centering
    \includegraphics[width=\linewidth]{LogBarrier2.png}
    \caption{Barrier functions}
    \label{fig:BarrierFunctions}
\end{figure}

Pink is the domain. The ellipsoids represent the shape of the hessian $\frac{\partial^2 J}{\partial \bo{x}^2}$ at different points on the domain. Green dot is $\bo{x}_a$.

\end{flushleft}
\end{frame}



\begin{frame}{Homework}
% \framesubtitle{Parameter estimation}
\begin{flushleft}

Visualize contours of a quadratic program of your choice.

\end{flushleft}
\end{frame}





\begin{frame}
\centerline{Lecture slides are available via Moodle.}
\bigskip
\centerline{You can help improve these slides at:}
\centerline{
\textcolor{blue}{\href{https://github.com/SergeiSa/Computational-Intelligence-Slides-Spring-2021}{github.com/SergeiSa/Computational-Intelligence-Slides-Spring-2021}}
}
\bigskip

\textcolor{black}{\qrcode[height=1.5in]{https://git.io/JYRBT}}
\bigskip

\centerline{Check Moodle for additional links, videos, textbook suggestions.}
\end{frame}



\end{document}
